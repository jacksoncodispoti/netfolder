\documentclass{article}
\usepackage[utf8]{inputenc}
\usepackage[english]{babel}
\usepackage[]{amsmath} %lets us use \begin{proof}
\usepackage[]{amsthm} %lets us use \begin{proof}
\usepackage[]{amssymb} %gives us the character \varnothing
\usepackage[]{multirow}
\usepackage[]{graphicx}

\title{CS371: Project Report}
\author{Jackson Codispoti}
\date\today
%This information doesn't actually show up on your document unless you use the maketitle command below

\begin{document}
\maketitle %This command prints the title based on information entered above

\section{Introduction}
\verb|netfolder| is a memory safe, Rust-programmed network folder program for uploading files to and downloading files from a shared network folder. 
\subsection{Building}
\subsection{Usage}

\section{Design}
For modularity purposes, this program is split into the following Rust modules: \verb|net|, \verb|stats|, \verb|encoding|, \verb|server|, and \verb|client|.
Most packet creation code, and packet parsing code is in the \verb|net| module.
The \verb|stats| module is used for collecting runtime stats such as upload speed, download speed, and other future metrics.
The \verb|encoding| module is used for handling the writing, reading, uploading, and downloading functionality of files.
The \verb|server| module contains the server code for handling requests.
The \verb|client| module contains the client code for initializing connections, executing command arguments, and for starting the user shell.
\subsection{net}
The purpose of this module is to handle a good amount of the networking code such as generating and parsing packets along with storing relevent constant variables.

\subsection{stats}
\subsection{encoding}
\subsection{server}
\subsection{client}

\section{Experiments}
\section{Conclusions}

\end{document}
